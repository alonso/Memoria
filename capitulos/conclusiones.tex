\chapter{Conclusiones}

En este trabajo se planteó el problema de comunicación anónima autentificada y se
demostró constructivamente que existe un protocolo que resuelve dicho problema.
Para ello se estudiaron tópicos avanzados de Criptografía como UC, GUC, Anonimato,
Desmentibilidad y distintas primitivas criptográficas asociadas a dichos tópicos.
Se definieron rigurosamente las propiedades que debe tener un protocolo para
resolver el problema planteado.
Se desarrolló un protocolo para el cual se puede garantizar matemáticamente que
satisface las propiedades necesarias para resolver el problema inicial.
El protocolo desarrollado es eficiente, pues se construye a partir de un protocolo
que se sabe eficiente por la comunidad \cite{Wikstrom04a} y adicionalmente efectúa
una operación que es lineal en el número de participantes del protocolo.\\

\section{Preguntas abiertas}

Adicionalmente, del trabajo realizado se desprenden las siguientes preguntas que han
quedado abiertas, pues escapan a los alcances de este trabajo. Sin embargo 
su sola formulación debe ser considerada un resultado de este trabajo, pues
estas preguntas proponen interesantes líneas de investigación.

\subsection{Mixnet GUC-segura}
El protocolo SIGMIX se construye a partir de una mixnet que en \cite{Wikstrom04a}
se demuestra UC-segura. Sin embargo, dado que para poder UC-realizar la mixnet
es necesario ocupar una setup assumption $\mathcal{F}_{KG}$, es válido preguntarse
si el protocolo de \cite{Wikstrom04a} sigue siendo seguro cuando $\mathcal{F}_{KG}$
es una funcionalidad compartida. En caso negativo es necesario preguntarse si existe
algún protocolo seguro para una mixnet con la setup assumption $\mathcal{F}_{KG}$.
Finalmente, si todos los esfuerzos han fracasado es necesario preguntarse si existe
una setup assumption compartida razonable con la cual se pueda GUC-realizar una
mixnet.\\

\subsection{Una modelación más exacta de la componibilidad en GUC}
El objetivo principal de GUC era modelar protocolos que se pueden componer
concurrentemente con otros protocolos que posiblemente comparten estado y de los
cuales nada se puede asumir. Sin embargo en el trabajo de esta memoria nos dimos
cuenta de que en la literatura, para el caso de PKI y CRS,
\footnote{En este caso se hace uso de uno funcionalidad compartida más fuerte
que CRS conocida como \textit{Augmented Common Random String}}
\textbf{sí se hacen suposiciones sobre los protocolos con los cuales el protocolo
analizado} se componen.\\
Para caso de PKI se asume que los protocolos
que hacen uso de la funcionalidad ideal $\bar{\mathcal{G}}_{KRK}$ se encuentran
dentro de un conjunto de protocolos $\Phi$. En general general $\Phi$ solo
contiene al protocolo analizado, por lo que la situación puede resultar muy
similar a JUC.\\
El caso de CRS es algo distinto, pero se puede mostrar que es equivalente. En
efecto, a la funcionalidad ideal que modela CRS se le especifica que debe
distinguir entre los participantes honestos y no honestos. Si bien en
\cite{Walfish:2008:ESM:1467461} notan que asumir esto es irreal, afirman que en
la práctica los participantes deben tener cuidado con no revelar cierto valor
secreto. Lo anterior no es más que una limitación a los protocolos que puede
ejecutar cada participante, pues $\Phi$ correspondería a protocolo que no revelan
el valor antes mencionado.\\
Adicionalmente, que solo protocolos en un conjunto $\Phi$
pueden acceder a la funcionalidad compartida no es posible de formalizar en EUC.
Si bien es posible de formalizar en GUC, es posible pensar en un modelo más general
donde $\Phi$ no depende de propiedades estáticas de los protocolos si no que de
propiedades dinámicas. Lo anterior no es formalizable en GUC, y para ellos es necesario
hacer ciertas modificaciones a GUC
