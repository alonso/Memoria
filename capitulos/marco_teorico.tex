%TODO Habrá un nombre mejor?
\chapter{Marco Teórico}

En este capítulo hacemos distintas definiciones básicas útiles
para este trabajo.

\section{Definiciones básicas}

\begin{definicion}[String]
Decimos que $\omega$ es un string si $\omega \in \{0,1\}^*$.
\end{definicion}

\begin{definicion}[Largo de un string]
Para un string $\omega$ denotamos por $|\omega|$ al entero $k$
tal que $k$ es el número de caracteres de $|\omega|$.
\end{definicion}

\begin{definicion}[Función despreciable]
Decimos que una función $\eta: \mathbb{N} \to \mathbb{N}$ es despreciable
si crece más lento que el inverso de cualquier polinomio. Es decir, para
todo polinomio $p$ existe
un $n_0$ talque para todo $n > n_0$ se cumple que
$$\eta(n) < \frac{1}{p(n)}$$
\end{definicion}
Las funciones despreciables permiten definir probabilidades ``pequeñas'' para cualquier
variable aleatoria ``algorítmica''. Es decir, variables aleatorias que corresponden a la
salida de un algoritmo.\\
Una forma de ver la utilidad de las funciones despreciables es la siguiente. Consideremos
el caso en que queremos determinar si un lenguaje $L$ es \textit{difícil}, esto es, para
cualquier algoritmo de tiempo polinomial determinar si $x \in L$ es difícil.
Diremos que un algorimo $A$ tiene éxito si $A(x) = 1$ cuando $x \in L$ y $A(x) = 0$
si no.
Si la probabilidad de éxito de un algoritmo $A$ de tiempo polinomial es depreciable, es imposible usar
$A$ como bloque de otro algoritmo de tiempo polinomial para obtener una mejor probabilidad de éxito.
Por el contrario, si la probabilidad de éxito de un algoritmo polinomial $A$ es pequeña pero no despreciable,
entonces es posible construir un algoritmo $B$ de tiempo polinomial cuya probabilidad de éxito es
$1-2^n$. 
\footnote{$B$ ejecuta $A$ como caja negra $t(n)$ veces y responde 1 si la mayoria
de las veces $A$ responde 1, y 0 en caso contrario. Es posible demostrar que existe un polinomio $q$ tal
que si $t(n) = q(n)$ la probabilidad de éxito de $B$ es $1-2^n$.}
De este modo podemos decir que $L$ es difícil si la probabilidad de éxito es depreciable para cualquier
algoritmo $A$; equivalentemente podemos decir que $L$ es fácil si existe un algoritmo $A$ con ventaja no
despreciable. Notemos que para el caso en que $L$ es fácil, la probabilidad de éxito del algoritmo $A$
puede ser muy pequeña, pero puede ser \textit{amplificada} hasta $1-2^n$, probabilidad para nada pequeña.

\begin{definicion}[Grupo Abeliano]
Decimos que el par $(G,+)$, donde $G$ es un conjunto y
$+:G^2 \to G$, es un Grupo Abeliano si se cumplen las siguientes propiedades:
\begin{enumerate}
\item (Asociatividad) $\forall a,b,c \in G$ $(a+b)+c=a+(b+c)$.
\item (Elemento neutro) $\exists 0 \in G$ tal que $\forall a \in G$ $a+0 = 0+a = a$.
\item (Inverso) $\forall a \in G$ $\exists -a \in G$ tal que $a+-a = 0$.
\item (Conmutatividad) $\forall a,b \in G$ $a+b = b+a$.
\end{enumerate}
\end{definicion}

En general, cuando decimos que $G$ es un grupo nos referimos a que es un Grupo Abeliano.

\begin{definicion}[Generador]
Decimos que $g\in G$ es un generador de $G$ si el conjunto generado por $g$ 
$\langle g \rangle = \{g^n|n\in\mathbb{N}\}$ es igual a $G$.
\end{definicion}

\begin{definicion}[Grupo cíclico]
Decimos que $G_q$ es un grupo cíclico de orden $q$ si existe un generador
$g$ de $G$ y $|G| = q$.

\end{definicion}

\section{Probabilidades discretas}

\begin{definicion}[Espacio de probabilidades finito]
Una espacio de probabilidades finito es un conjunto finito
$\Omega = \{\omega_1, \ldots, \omega_{|\Omega|}\}$ con un conjunto
de números $p_1, \ldots, p_{|\Omega|} \in [0,1]$ tales que
$\sum_{i=1}^{|\Omega|}p_i=1$.
\end{definicion}

A menos que se especifique lo contrario, asumimos que la distribución
de $\Omega$ es uniforme, es decir $p_i = \frac{1}{|\Omega|}$. Al experimento
de escoger un $\omega$ en $\Omega$ al azar lo denotamos por $\omega \in_R \Omega$.

\begin{definicion}[Evento]
Decimos que $A$ es un evento de $\Omega$ si $A \subseteq \Omega$. Denotamos
por $\Pr[A] = \sum_{\omega_i \in A}p_i$  a la probabilidad de $A$.
\end{definicion}

\begin{definicion}[Variable aleatoria]
Una variable aleatoria es una función $X:\Omega \to \mathbb{R}$ que asigna a
cada $\omega \in \Omega$ un número real $x \in \mathbb{R}$.
\end{definicion}

\begin{definicion}[Distribución]
Dada una variable aleatoria, definimos su distribución como la función
$f:\mathbb{R} \to [0,1]$ tal que $f(x) = \sum_{i:X(\omega_i)=x}p_i$.
\end{definicion}

\begin{definicion}[Distancia estadística]
Dadas dos variables aleatorias $X$ e $Y$ definidas sobre $\Omega$, definimos
la distancia estadística entre ellas como
$$\Delta(X, Y) = \max_{S\subseteq \Omega}\{X(S)-Y(S)\}$$.
\end{definicion}

\subsection{Indistinguibilidad}
La noción de indistinguibilidad es una noción fundamental en criptografía
pues ha permitido definir la seguridad mediante la similitud entre distintos
eventos.\\
La indistinguibilidad es una noción
de similaridad entre familias de variables aleatorias (o conjuntos
de variables aleatorias) y existen tres tipos de indistinguibilidad.

\begin{definicion}[Indistinguibilidad perfecta]
Decimos que dos familias de variables aleatorias $U=\{U_x\}_{x\in\{0,1\}^*}$
y $V=\{V_x\}_{x\in\{0,1\}^*}$ son perfectamente indistinguibles
si para todo $x$ $\Delta(U_x, V_x) = 0$.
Denotamos la indistinguibilidad perfecta por $U = V$.
\end{definicion}

\begin{definicion}[Indistinguibilidad estadística]
Decimos que dos familias de variables aleatorias $U=\{U_x\}_{x\in\{0,1\}^*}$ y 
$V=\{V_x\}_{x\in\{0,1\}^*}$ son estadísticamente indistinguibles
si $\Delta(U_x, V_x)$ es despreciable en $|x|$. Denotamos la indistinguibilidad estadística
por $U \approx Y$.
\end{definicion}

\begin{definicion}[Indistinguibilidad computacional]
Decimos que dos familias de variables aleatorias $U=\{U_x\}_{x\in\{0,1\}^*}$ y
$V=\{V_x\}_{x\in\{0,1\}^*}$ son computacionalmente indistinguibles si
para todo algoritmo polinomial aleatorizado $A$ existe una función despreciable
$\eta$ y un entero $n_0$ talque si $|x| > n_0$
$|\Pr[A(U_x)=1] - \Pr[A(V_x)=1]| \leq \eta(|x|)$. Denotamos la indistinguibilidad
computacional por $U \overset{c}{\approx} V$.
\end{definicion}

\section{Definiciones criptográficas básicas}

\subsection{El problema de Diffie-Hellman decisional (DDH)}
El problema DDH \cite{diffie-hellman_me:1976a} es un problema que se considera
difícil, es decir se conjetura
que no existe un algoritmo a tiempo polinomial que lo solucione para ciertos
grupos. DDH es una herramienta fundamental en la Criptografía moderna pues
permite la construcción eficiente de primitivas con una variada aplicabilidad.

\begin{definicion}[DDH]
Sea $G_q$ un grupo cíclico de orden $q$ y $g$ un generador de $G_q$. Decimos que en $G_q$ se
cumple DDH si $(g^\alpha, g^\beta, g^\gamma) \overset{c}{\approx}
(g^\alpha, g^\beta, g^{\alpha \cdot \beta})$ dado que $\alpha, \beta, \gamma \in_R G_q$.
\end{definicion}

\subsection{Esquemas de Encriptación}
Sea $\mathcal{E} = (K, E, D)$, donde $K:\strings\times\strings \to \strings$ es el
algoritmo de generación de claves; $E: M\times \strings \to C$ es el algoritmo de encriptación;
$D: C \times \strings \to M$ es el algoritmo de desencriptación; $M$ es el conjunto de textos
planos; y $C$ es el conjunto de textos cifrados. Decimos que $\mathcal{e}$ es un esquema de
encriptación simétrico si para todo string aleatorio
$r \in \{0, 1\}^\kappa$ con $\kappa$ el parámetro de seguridad, dado $k = K(\kappa; r)$
(es decir para toda clave generada al azar), todo texto plano $m\in M$ puede ser recuperado
correctamente, esto es $D_k(E_k(m))=m$.\\

Decimos que $\mathcal{E}$ es un esquema de encriptación asimétrico si para todo
$r \in \{0, 1\}^\kappa$, dado $(sk, pk) = K(\kappa; r)$, para todo texto plano $m\in M$
$D_{sk}(E_{pk}(m))=m$. $pk$ es la clave pública (todos los participantes la conocen) y $sk$
es la clave privada (solo un participante la conoce).


\subsubsection{Indistinguibilidad bajo ataque de texto plano escogido (IND-CPA)}
También conocida como \textit{seguridad semántica} es una definición de privacidad de un esquema
criptográfico. Apunta a que un esquema de encriptación $\mathcal{E} = (K, E, D)$ es seguro si ningún
atacante es capaz de distinguir entre la encriptación de cualesquier par de mensajes $m_1, m_2$.


\begin{definicion}[Experimento IND-CPA]
Sea $\mathcal{A}$ un algoritmo, $\mathcal{E}$ un esquema de encriptación, $\kappa$ el parámetro de
seguridad y $b$ un bit. ,
Definimos el experimento IND-CPA, denotado por $\mathrm{EXP}^\mathrm{IND-CPA}_\mathcal{E}(\mathcal{A}, b, \kappa)$,
como la variable aleatoria resultante de
ejecutar el experimento descrito en la Figura \ref{fig:ind_cpa}
\end{definicion}

En el experimento IND-CPA basicamente se genera una clave al azar y se le permite a un adversario $A$
hacer preguntas a un \textit{oráculo} que responde el texto cifrado correspondiente a la encriptación de uno
(fijo) de los textos planos pasados como argumentos al oráculo. El objetivo de $A$ es determinar el bit $b$,
es decir, cúal de los argumentos es el que encripta el oráculo. Notemos que $A$ puede obtener encriptaciones
de los textos planos que desee, haciendo preguntas de la forma $(m,m)$. De ahí el nombre \textit{texto plano
escogido}, pues se le permite al adversario obtener las encriptaciones de textos planos escogidos por él.

\begin{definicion}
Definimos la ventaja IND-CPA de $\mathcal{A}$, denotada por $\advindcpa{\encscm}{\adv}{\kappa}$,
como sigue
$$
\advindcpa{\encscm}{\adv}{\kappa} =
\abs
{
    \mathrm{EXP}^\mathrm{IND-CPA}_\mathcal{E}(\mathcal{A}, b, \kappa)
    -
    \mathrm{EXP}^\mathrm{IND-CPA}_\mathcal{E}(\mathcal{A}, b, \kappa)
}.
$$
\end{definicion}

\begin{definicion}[Seguridad IND-CPA]
Decimos que un esquema de encriptación $\mathcal{E}$ es IND-CPA seguro si
para todo adversario $\adv$ $\advindcpa{\encscm}{\adv}{\kappa}$ es una función
despreciable en $\kappa$.
\end{definicion}

\begin{figure}
\begin{centering}
\framebox{\begin{minipage}[t]{1\columnwidth}
El experimento IND-CPA  ejecutado con adversario $\mathcal{A}$, esquema de encriptación $\mathcal{E} = (K, E, D)$,
un bit $b \in\{0,1\}$ y parámetro de seguridad $\kappa$ procede como sigue:
\begin{enumerate}
    \item Escoger las clave del esquema, $k \overset{R}{\leftarrow} K(\kappa)$.
    \item Ejecutar al adversario $\mathcal{A}$ y cada vez que escriba $(m_1, m_2)$ retornale $E_k(m_b)$.
    \item Retornar lo que $\mathcal{A}$ retorne.
\end{enumerate}
\end{minipage}}
\end{centering}
\caption{El experimento IND-CPA}
\label{fig:ind_cpa}
\end{figure}

\subsection{Esquemas de autentificación de mensajes}
\label{sect:uf-cma}
Decimos que $\mathcal{MA}=(K,T,V)$, con $M$ el conjunto de mensajes $\{0,1\}^\kappa$ el conjunto de claves,
es un esquema de autentificación de
mensajes si $K$ es un algoritmo de generación de claves y para todo $m \in M$ y todo
$k \in \{0,1\}^\kappa$ se tiene que $v(m, T_k(m)) = 1$.

\subsubsection{Infalsificabilidad ante ataques de mensajes escogidos UF-CMA}
UF-CMA es una noción de seguridad que aplica a esquemas de autentificación de mensajes
\footnote{A veces, si $\mathcal{MA} = (K, T, V)$ y $T$ corresponde a un algoritmo MAC,
con MAC nos referimos indistintamente al algoritmo $T$ y al esquema $\mathcal{MA}$.}.
Un esquema de firmas
es UF-CMA si ningún adversario es capaz de falsificar la firma de un mensaje. Formalizamos esta noción
con las siguientes definiciones.

\begin{definicion}[Ventaja UF-CMA]
Definimos la ventaja \textrm{UF-CMA} de un adversario $\mathcal{A}$ con el esquema $\mathcal{MA} = (K, T, V)$
como:
$$\mathrm{Adv}^\text{UF-CMA}_{\mathcal{MA},\mathcal{A}}(\kappa) =
\Pr[\text{UF-CMA}_{\mathcal{MA},\mathcal{A}}(\kappa) = 1]$$
donde $\text{UF-CMA}_{\mathcal{MA},\mathcal{A}}(\kappa)$ es la variable aleatoria obtenida al
ejecutar el experimento de la figura \ref{fig:uf-cma}.
\end{definicion}

\begin{definicion}[Seguridad UF-CMA]
Decimos que un esquema de autentificación de mensajes $\mathcal{MA}$ es UF-CMA seguro si para todo
adversario polinomial aleatorizado $\mathcal{A}$ existe una función despreciable $\eta$ y un $\kappa_0$
talque para todo $\kappa > \kappa_0$
$$\mathrm{Adv}^{UF-CMA}_{\mathcal{MA}, \mathcal{A}}(\kappa) \leq \eta(\kappa)$$
\end{definicion}

\begin{figure}
\framebox{\begin{minipage}[t]{1\columnwidth}
El experimento UF-CMA  ejecutado con adversario $\mathcal{A}$, esquema de autentificación de mensajes
$\mathcal{MA} = (K, T, V)$ y parámetro de seguridad $\kappa$ procede como sigue:
\begin{enumerate}
    \item Escoger la clave del esquema, $k \overset{R}{\leftarrow} K(\kappa)$ e inicializar
          $\Gamma \leftarrow \emptyset$.
    \item Ejecutar al adversario $\mathcal{A}$ y cada vez que escriba $m$ retornale $T_k(m)$ y
          actualizar $\Gamma \leftarrow \Gamma \cup \{m\}$.
    \item Cuando $\mathcal{A}$ escriba $(m, t)$, si $m \notin \Gamma$ y $T_k(m) = t$ retornar 1
          y 0 de lo contrario.
\end{enumerate}
\end{minipage}}
\caption{El experimento UF-CMA}
\label{fig:uf-cma}
\end{figure}

\subsection{Supuestos de inicialización}
Los supuestos de inicialización son necesarios para implementar ciertos protocolos criptográficos
que de otra forma serían imposible de realizar. Los supuestos de inicialización corresponden a funcionalidades
ejecutadas por una entidad confiable, y en UC son representados por una funcionalidad ideal. A continuación
listamos dos supuestos de inicialización.


\subsubsection{String público aleatorio CRS}
También conocido como string público de referencia, corresponde a una entidad confiable que publica un string
con una distribución fija. En general se considera que la distribución del string es una uniforme.\\
En UC se modela CRS con la funcionalidad ideal $\mathcal{F}_{CRS}$ descrita en la figura \ref{fig:crs}i

\begin{figure}
\framebox{\begin{minipage}[t]{1\columnwidth}
La funcionalidad ideal $\mathcal{F}_{CRS}^D$ parametrizada por una distribución $D$ procede como sigue:
\begin{enumerate}
    \item Si recibe $(\mathtt{CRS})$ de $P$, verificar que $P\in\mathcal{P}$, donde $\mathcal{P}$ es un conjunto
          de identidades, si $P \notin \mathcal{P}$ ignorar a $P$.
    \item Si no hay un valor $r$ registrado escoger y registrar $r \overset{R}{\leftarrow} D$.
    \item Enviar $(\mathtt{CRS}, r)$ a $P$.
\end{enumerate}
\end{minipage}}
\caption{La funcionalidad ideal $\mathcal{F}_{CRS}$}
\label{fig:crs}
\end{figure}


\subsubsection{Interfaz de clave pública PKI}
Una interfaz de clave pública o PKI (del inglés \textit{Public Key Interface}) es un supuesto de inicialización
que permite registrar pares de llaves públicas/privadas en una base de datos confiable y autentificada.\\
En GUC una forma de modelar PKI es con la funcionalidad compartida $\bar{\mathcal{G}}_{KRK}$ definida en la figura
\ref{fig:krk}. Notamos que $\bar{\mathcal{G}}_{KRK}$ viene parametrizada por un conjunto $\Phi$, que corresponde
a los protocolos a los cuales les es permitido obtener claves privadas. 

\begin{figure}
\framebox{\begin{minipage}[t]{1\columnwidth}
La funcionalidad compartida $\bar{\mathcal{G}}_{KRK}$ parametrizada por una conjunto de protocolos permitidos
$\Phi$, un algoritmo de generación de claves $\mathtt{Gen}$ y parámetro de seguridad $\kappa$,
procede como sigue:
\begin{enumerate}
    \item Si recibe un mensaje $(\mathtt{register})$ de un participante honesto $P_i$ que aún no se ha registrado
          escoger $r \overset{R}{\leftarrow} \{0,1\}^\kappa$, luego calcular
          $(pk_i, sk_i) \leftarrow \mathtt{Gen}^\kappa (r)$ y guardar la tupla $(P_i, pk_i, sk_i)$.
    \item Si recibe un mensaje $(\mathtt{register}, r)$ de un participante corrupto $P_i$ que aún no se ha registrado
          calcular
          $(pk_i, sk_i) \leftarrow \mathtt{Gen}^\kappa (r)$ y guardar la tupla $(P_i, pk_i, sk_i)$.
    \item Si recibe un mensaje $(\mathtt{retrieve}, P_i)$ de $P_j$, si existe una tupla registrada
          $(P_i, pk_i, sk_i)$ retornar $(P_i, pk_i)$. De lo contrario retornar $(P_i, \perp)$.
    \item Si recibe un mensaje $(\mathtt{retrievesecret}, P_i)$ de $P_j$, si $i = j$, $P_j$ es corrupto
          esta corriendo un código en $\Phi$ y existe una tupla registrada
          $(P_i, pk_i, sk_i)$ retornar $(P_i, sk_i)$. De lo contrario retornar $(P_i, \perp)$.
\end{enumerate}
\end{minipage}}
\caption{La funcionalidad compartida $\bar{\mathcal{G}}_{KRK}$}
\label{fig:krk}
\end{figure}

