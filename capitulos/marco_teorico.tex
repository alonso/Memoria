\chapter{Modelo Criptográfico}

Demostraremos las garantías de seguridad de nuestro protocolo en el
\textit{framework Generalized Universal Composability} (GUC). GUC \cite{conf/tcc/CanettiDPW07}
es una generalizacion del \textit{framework Universal Composability}
\cite{conf/focs/Canetti01}. Ambos sirven para modelar protocolos criptgráficos concurrentes,
pero GUC adicionalmente modela protocolos concurrentes que comparten estado entre si.
Primero revisaremos el framework UC pues GUC se construye a partir de UC con pequeñas,
pero muy significativas, modificaciones.\\

\subsection{The Universal Composability framework (UC)}
The UC framework is a methodology for a modularized designing and modularized proving the security of
\textit{real world} cryptographics protocols. The spirit of UC is to create a protocol and then abstract the
security one  want the protocol have into a idealized secure protocol. Then it must be shown that
executing the real or the idealized protocol is essentially the same, and then real protocol is as
secure as the idealized protocol.\\
Real world protocols are modeled as protocols running in a complete concurrent setting with many other protocols and
possibly distributed among many parties. As in the real world the protocol can be monitored and some participants
can have ``undesired" behavior, attempting with the desired security of the protocol. All the possible undesired behavior
executed by one machine, the real adversary $\mathcal{A}$. In an execution of the protocol the adversary can eavesdrop and
manipulate all the data sent from one  party to another, can corrupt some parties and then execute
arbitrary code on them. On the other hand the ideal protocol, called the ideal functionality usually denoted by
$\mathcal{F}$, runs in an ideal setting,
that is executed by a secure party and the monitoring is restricted. In the ideal world there exists an ideal adversary
$\mathcal{S}$ but it cannot
eavesdrop and is only allowed to stop or drop exchanged data. The setting where the ideal protocol is executed with the
real adversary is known as the \textit{real world}, and the setting where the ideal protocol is executed with the ideal
adversary is known as the \textit{ideal world}. In both worlds concurrent execution with arbitrary other protocols is
given by a special machine known as environment. In UC both the real an ideal protocol are analyzed isolated from other
possibly concurrent protocols. It is considered that all the externals protocols and the outside calls to the protocol
are are executed by the environment.
To ensure that the real protocol achieves
the desired security it must hold that that for any execution of the protocol in real world and for any adversarial
strategy (that is for all environments and all real adversaries) there exists an adversarial strategy with limited
resources (an ideal adversary) that has the same effect (the environment is not aware of any difference between the
two strategies).\\


\section{Generalized Universal Composabillity}

Before the GUC framework were proposed an alternative way
to model protocols sharing state in the UC framework was the use of JUC theorem, but this is not as general as
it only allows protocols sharing state among themselves. Instead the GUC framework models protocols sharing state with
unpredictable other protocols.\\

