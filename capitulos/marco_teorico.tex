%TODO Habrá un nombre mejor?
\chapter{Marco Teórico}

En este capítulos hacemos distintas definiciones básicas útiles
para este trabajo.

\section{Definiciones básicas}

\begin{definicion}[String]
Decimos que $\omega$ es un string si $\omega \in \{0,1\}^*$.
\end{definicion}

\begin{definicion}[Largo de un string]
Para un string $\omega$ denotamos por $|\omega|$ al entero $k$
talque $k$ es el número de caracteres de $|\omega|$.
\end{definicion}

\begin{definicion}[Función despreciable]
Decimos que una función $\eta: \mathbb{N} \to \mathbb{N}$ es despreciables
si crece más lento que el inverso de cualquier polinomio. Es decir, para
todo polinomio $p$ existe
un $n_0$ talque para todo $n > n_0$ se cumple que
$$\eta(n) < \frac{1}{p(n)}$$
\end{definicion}
Las funciones despreciables permiten definir probabilidades ``pequeñas'' cualquier
variable aleatorias ``algorítmicas'' (es decir corresponden a las salida de un algoritmo).
La definición busca eliminar el efecto de ejecutar una cantidad polinomial de veces un
algoritmo (usar el lema de amplificación).

\begin{definicion}[Grupo Abeliano]
Decimos que el par $(G,+)$, donde $G$ es un conjunto y
$+:G^2 \to G$, es un grupo abeliano si se cumplen las siguientes propiedades:
\begin{enumerate}
\item (Asocioatividad) $\forall a,b,c \in G$ $(a+b)+c=a+(b+c)$.
\item (Elemento neutro) $\exists 0 \in G$ talque $\forall a \in G$ $a+0 = 0+a = a$.
\item (Inverso) $\forall a \in G$ $\exists -a \in G$ talque $a+-a = 0$.
\item (Conmutatividad) $\forall a,b \in G$ $a+b = b+a$.
\end{enumerate}
\end{definicion}

En general cuando decimos que $G$ es un grupo nos referimos a que es un grupo abeliano.

\begin{definicion}[Generador]
Decimos que $g\in G$ es un generador de $G$ si el conjunto generado por $g$ 
$\langle g \rangle = \{g^n|n\in\mathbb{N}\}$ es igual a $G$
\end{definicion}

\begin{definicion}[Grupo cíclico]
Decimos que $G_q$ es un grupo cíclico de orden $q$ si existe un generador
$g$ de $G$ y $|G| = q$.

\end{definicion}

\section{Probabilidades discretas}

\begin{definicion}[Espacio de probabilidades finito]
Una espacio de probabilidades finito es un conjunto finito
$\Omega = \{\omega_1, \ldots, \omega_{|\Omega|}\}$ con un conjunto
de números $p_1, \ldots, p_{|\Omega|} \in [0,1]$ tales que
$\sum_{i=1}^{|\Omega|}p_i=1$
\end{definicion}

A menos que se especifique lo contrario, asumimos que la distribución
de $\Omega$ es uniforme, es decir $p_i = \frac{1}{|\Omega|}$. Al experimento
de escoger un $\omega$ en $\Omega$ al azar denotamos por $\omega \in_R \Omega$.

\begin{definicion}[Evento]
Decimos que $A$ es un evento de $\Omega$ si $A \subseteq \Omega$. Denotamos
por $\Pr[A] = \sum_{\omega_i \in A}p_i$  a la probabilidad de $A$
\end{definicion}

\begin{definicion}[Variable aleatoria]
Una variable aleatoria es una función $X:\Omega \to \mathbb{R}$ que asigna a
cada $\omega \in \Omega$ un número real $x \in \mathbb{R}$.
\end{definicion}

\begin{definicion}[Distribución]
Dada una varibale aleatoria, definimos su distribución como la función
$f:\mathbb{R} \to [0,1]$ talque $f(x) = \sum_{i:X(\omega_i)=x}p_i$
\end{definicion}

\begin{definicion}[Distancia estadística]
Dadas dos variables aleatoria $X$ e $Y$ definidas sobre $\Omega$, definimos
la distancia estadística entre ellas como
$$\Delta(X, Y) = \max_{S\subseteq \Omega}\{X(S)-Y(S)\}$$
\end{definicion}

\subsection{Indistinguibilidad}
La noción de indistinguibilidad es una noción fundamental en criptografía
pues ha permitido definir la seguridad mediante la similitud entre distintos
eventos.\\
La indistinguibilidad es una noción
de similaridad entre familias de variables aleatorias (o conjuntos
de variables aleatorias) y existen tres tipos de indistinguibilidad.

\begin{definicion}[Indistinguibilidad perfecta]
Decimos que dos familias de variables aleatorias $U=\{U_x\}_{x\in\{0,1\}^*}$
y $V=\{V_x\}_{x\in\{0,1\}^*}$ son perfectamente indistinguibles
si $\Delta(U_x, V_x)) = 0 \qquad \forall x$.
Denotamos la indistinguibilidad perfecta por escribimos $U_x = V_x$
\end{definicion}

\begin{definicion}[Indistinguibilidad estadística]
Decimos que dos familias de variables aleatorias $U=\{U_x\}_{x\in\{0,1\}^*}$ y 
$V=\{V_x\}_{x\in\{0,1\}^*}$ son estadísticamente indistinguibles
si $\Delta(U_x, V_x)$ es despreciable en $|x|$. Denotamos la indistinguibilidad estadística
por $U \approx Y$.
\end{definicion}

\begin{definicion}[Indistinguibilidad computacional]
Decimos que dos familias de variables aleatorias $U=\{U_x\}_{x\in\{0,1\}^*}$ y
$V=\{V_x\}_{x\in\{0,1\}^*}$ son estadísticamente indistinguibles
para todo algoritmo polinomial aleatorizado $A$ existe una función despreciable
$\eta$ y un entero $n_0$ talque si $|x| > n_0$
$|\Pr[A(U_x)=1] - \Pr[A(V_x)=1]| \leq \eta(|x|)$. Denotamos la indistinguibilidad
computacional por $U \overset{c}{\approx} V$.
\end{definicion}

\section{Definiciones criptográficas básicas}

\subsection{El problema de Diffie-Hellman decisonal (DDH)}
El problema DDH \cite{diffie-hellman_me:1976a} es un problema que se considera
difícil, es decir se conjetura
que no existe un algoritmo polinomial que lo solucione, para ciertos
grupos. DDH es una herramienta fundamental en la Criptografía moderna pues
permite la construcción eficientes primitivas con una variada aplicabilidad.

\begin{definicion}[DDH]
Sea $G_q$ un grupo cíclo de orden $q$. Decimos que en $G_q$ se cumple DDH
si las variables aleatorias $(g^\alpha, g^\beta, g^\gamma) \overset{c}{\approx}
(g^\alpha, g^\beta, g^{\alpha \cdot \beta})$ dado que $\alpha, \beta, \gamma \in_R G_q$.
\end{definicion}

\subsection{Esquemas de Encriptación}
Decimos que $\mathcal{E} = (K, E, D)$, con conjunto de textos planos $M$ y conjunto de textos
cifrados $C$,es un esquema de encriptación simétrico si para todo
$r \in \{0, 1\}^\kappa$, dado $k = K(\kappa; r)$, para todo $m\in M$ $D_k(E_k(m))=m$.

Decimos que $\mathcal{E} = (K, E, D)$, con conjunto de textos planos $M$ y conjunto de textos
cifrados $C$,es un esquema de encriptación asimétrico si para todo
$r \in \{0, 1\}^\kappa$, dado $(sk, pk) = K(\kappa; r)$, para todo $m\in M$ $D_{sk}(E_{pk}(m))=m$.


\subsubsection{Indistinguibilidad bajo ataque de texto plano escogido (IND-CPA)}
Tabién conocida como \textit{seguridad semántica} es una definición de privacidad de un esquema
critográfico. Apunta a que un esquema de encriptación $\mathcal{E} = (K, E, D)$ es seguro si ningún atacante es capaz de
dinstinguir entre la encriptación de cualesqueir par de mensajes $m_1, m_2$.


\begin{definicion}[Experimento IND-CPA]
Definimos el experimento IND-CPA parametrizado por un aloritmo $\mathcal{A}$,
un esquema de encriptación $\mathcal{E}$, parametro de seguridad $\kappa$ y
un bit $b$, denotado por $\mathrm{EXP}^\mathrm{IND-CPA}_\mathcal{E}(\mathcal{A}, b)$,
como la variable aleatoria resultante de
ejecutar el experimento descrito en la figura \ref{fig:ind_cpa}
\end{definicion}


\begin{definicion}[Seguridad IND-CPA]
Decimos que un esquema de encriptación $\mathcal{E}$ es IND-CPA seguro si
$$\mathrm{EXP}^\mathrm{IND-CPA}_\mathcal{E}(\mathcal{A}, 1) \approx
\mathrm{EXP}^\mathrm{IND-CPA}_\mathcal{E}(\mathcal{A}, 0)$$
\end{definicion}

\begin{figure}
\begin{centering}
\framebox{\begin{minipage}[t]{1\columnwidth}
El experimento IND-CPA  ejecutado con adversario $\mathcal{A}$, esquema de encritación $\mathcal{E} = (K, E, D)$,
un bit $b \in\{0,1\}$ y parámetro de seguridad $\kappa$ procede como sigue:
\begin{enumerate}
    \item Escoger las clave del esquema, $k \overset{R}{\leftarrow} K(\kappa)$.
    \item Ejecutar al adversario $\mathcal{A}$ y cada vez que escriba $(m_1, m_2)$ retornale $E_k(m_b)$.
    \item Retornar lo que $\mathcal{A}$ retorne.
\end{enumerate}
\end{minipage}}
\end{centering}
\caption{El experimento IND-CPA}
\label{fig:ind_cpa}
\end{figure}

\subsection{Esquemas de autentificación de mensajes}
\label{sect:uf-cma}
Decimos que $\mathcal{MA}=(K,T,V)$, con $M$ el conjunto de mensajes $\{0,1\}^\kappa$ el conjunto de claves,
es un esquema de autentificación de
mensajes si $K$ es un algoritmo de generación de claves y para todo $m \in M$ y todo
$k \in \{0,1\}^\kappa$ se tiene que $v(m, T_k(m)) = 1$.

\subsubsection{Infalsificabilidad ante ataques de mensajes escogidos UF-CMA}
UF-CMA es una noción de seguridad que aplica a esquemas de autentificación de mensajes
\footnote{A veces, si $\mathcal{MA} = (K, T, V)$ y $T$ corresponde a un algoritmo MAC,
con MAC nos referimos indistintamente al algoritmo $T$ y al esquema $\mathcal{MA}$.}.
Un esquema de firmas
es UF-CMA si ningún adversario es capaz de falsificar la firma de un mensaje. Formalizamos esta noción
con las siguientes definiciones.

\begin{definicion}[Ventaja UF-CMA]
Definimos la ventaja \textrm{UF-CMA} de un adverario $\mathcal{A}$ con el esquema $\mathcal{MA} = (K, T, V)$
como:
$$\mathrm{Adv}^\text{UF-CMA}_{\mathcal{MA},\mathcal{A}}(\kappa) =
\Pr[\text{UF-CMA}_{\mathcal{MA},\mathcal{A}}(\kappa) = 1]$$
donde $\text{UF-CMA}_{\mathcal{MA},\mathcal{A}}(\kappa)$ es la variable aleatoria obtenida al
ejecutar el experimento de la figura \ref{fig:uf-cma}.
\end{definicion}

\begin{definicion}[Seguridad UF-CMA]
Decimos que un esquema de autentificación de mensajes $\mathcal{MA}$ es UF-CMA seguro si para todo
adversario polinomial aleatoriazdo $\mathcal{A}$ existe una función despreciable $\eta$ y un $\kappa_0$
talque para todo $\kappa > \kappa_0$
$$\mathrm{Adv}^{UF-CMA}_{\mathcal{MA}, \mathcal{A}}(\kappa) \leq \eta(\kappa)$$
\end{definicion}

\begin{figure}
\framebox{\begin{minipage}[t]{1\columnwidth}
El experimento UF-CMA  ejecutado con adversario $\mathcal{A}$, esquema de autetificación de mensajes
$\mathcal{MA} = (K, T, V)$ y parámetro de seguridad $\kappa$ procede como sigue:
\begin{enumerate}
    \item Escoger la clave del esquema, $k \overset{R}{\leftarrow} K(\kappa)$ e inicializar
          $\Gamma \leftarrow \emptyset$.
    \item Ejecutar al adversario $\mathcal{A}$ y cada vez que escriba $m$ retornale $T_k(m)$ y
          actualizar $\Gamma \leftarrow \Gamma \cup \{m\}$.
    \item Cuando $\mathcal{A}$ escriba $(m, t)$, si $m \notin \Gamma$ y $T_k(m) = t$ retornar 1
          y 0 de lo contrario.
\end{enumerate}
\end{minipage}}
\caption{El experimento UF-CMA}
\label{fig:uf-cma}
\end{figure}

\subsection{Supestos de incialización}
Los supuestos de incialización son necesarios para implementar ciertos protocolos criptográficos
que de otra forma serían imposible de realizar. Los supuestos de incialización corresponden a funcionalidades
ejecutadas por una entidad confiable, y en UC son representados por una funcionalidad ideal. A continuación
listamos dos supuestos de inicialización.


\subsubsection{String público aleatorio CRS}
También conocido como string público de referencia, corresponde a una entidad confiable que publica un string
con una distribución fija. En general se considera que la distribución del string es una uniforme.\\
En UC se modela CRS con la funcionalidad ideal $\mathcal{F}_{CRS}$ descrita en la figura \ref{fig:crs}i

\begin{figure}
\framebox{\begin{minipage}[t]{1\columnwidth}
La funcionalidad ideal $\mathcal{F}_{CRS}^D$ parametrizada por una distribución $D$ procede como sigue:
\begin{enumerate}
    \item Si recibe $(\mathtt{CRS})$ de $P$, verificar que $P\in\mathcal{P}$, donde $\mathcal{P}$ es un conjunto
          de identidades, si $P \notin \mathcal{P}$ ignorar a $P$.
    \item Si no hay un valor $r$ registrado escoger y registrar $r \overset{R}{\leftarrow} D$.
    \item Enviar $(\mathtt{CRS}, r)$ a $P$.
\end{enumerate}
\end{minipage}}
\caption{La funcionalidad ideal $\mathcal{F}_{CRS}$}
\label{fig:crs}
\end{figure}


\subsubsection{Interfaz de clave pública PKI}
Una interfaz de clave pública o PKI (del inglés \textit{Public Key Interface}) es un supuesto de inicialización
que permite registrar pares de llaves públicas/provadas en una base de datos confiable y autentificada.\\
En GUC una forma de modelar PKI es con la funcionalidad compartida $\bar{\mathcal{G}}_{KRK}$ definida en la figura
\ref{fig:krk}. Notamos que $\bar{\mathcal{G}}_{KRK}$ viene parametrizada por un conjunto $\Phi$, que corresponde
a los protocolos a los cuales les es permitido obtener claves privadas. 

\begin{figure}
\framebox{\begin{minipage}[t]{1\columnwidth}
La funcionalidad compartida $\bar{\mathcal{G}}_{KRK}$ parametrizada por una conjunto de protocolos permitidos
$\Phi$, un algoritmo de generación de claves $\mathtt{Gen}$ y parámetro de seguridad $\kappa$,
procede como sigue:
\begin{enumerate}
    \item Si recibe un mensaje $(\mathtt{register})$ de un participante honesto $P_i$ que aún no se ha registrado
          escoger $r \overset{R}{\leftarrow} \{0,1\}^\kappa$, luego calcular
          $(pk_i, sk_i) \leftarrow \mathtt{Gen}^\kappa (r)$ y guardar la tupla $(P_i, pk_i, sk_i)$.
    \item Si recibe un mensaje $(\mathtt{register}, r)$ de un participante corrupto $P_i$ que aún no se ha registrado
          calcular
          $(pk_i, sk_i) \leftarrow \mathtt{Gen}^\kappa (r)$ y guardar la tupla $(P_i, pk_i, sk_i)$.
    \item Si recibe un mensaje $(\mathtt{retrieve}, P_i)$ de $P_j$, si existe una tupla registrada
          $(P_i, pk_i, sk_i)$ retornar $(P_i, pk_i)$. De lo contrario retornar $(P_i, \perp)$.
    \item Si recibe un mensaje $(\mathtt{retrievesecret}, P_i)$ de $P_j$, si $i = j$, $P_j$ es corrupto
          esta corriendo un código en $\Phi$ y existe una tupla registrada
          $(P_i, pk_i, sk_i)$ retornar $(P_i, sk_i)$. De lo contrario retornar $(P_i, \perp)$.
\end{enumerate}
\end{minipage}}
\caption{La funcionalidad compartida $\bar{\mathcal{G}}_{KRK}$}
\label{fig:krk}
\end{figure}

