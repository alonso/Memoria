\chapter{Desmentibilidad}
Deniable authentication was first defined, outside the (G)UC framework, by Dwork, Nahor and Sahai in
\cite{DwoNaoSah04}. Several modifications and generalizations have been made since then, and here we consider
the definition given by Dodis, Katz, Smith and Walfish in \cite{conf/tcc/DodisKSW09}, since that is the one who
applies to a concurrent and distributed setting\\
Roughly speaking a protocol is deniable if nobody can prove that a particular session of
the protocol is taking place or have ever took place. In \cite{conf/tcc/DodisKSW09} is shown that such property
can be achieved considering an on line judge who must decide who is he talking to: an informant who is observing
a real session of the authentication protocol, or a mis informant who do not have access to the real session of
the protocol but still try to convince the judge that the session is taking place. The protocol is said to be
an on-line deniable authentication protocol if for all judge and all informants there exist a mis informant such that
the judge can't distinguish from the informant and the mis informant with overwhelming probability. In the
full version of \cite{conf/tcc/DodisKSW09} is demonstrated that this notion is equivalent to GUC-realize the
functionality $\mathcal{F}_{auth}$. They pointed out that in the GUC framework a protocol GUC-realizing
a functionality $\mathcal{F}$ is as deniable as $\mathcal{F}$. The ideal functionality $\mathcal{F}_{auth}$ is
``fully simulatable'', that means that the protocol can be completely simulated without the participation
% Tengo una duda. En conf/tcc/DodisKSW09 nunca definen a los protocolos on-line deniable, solo definen
% los on-line deniable authentication. No se como definir los on-line deniable sin que paresca que los
% defini yo, porque salvo la definicion con titulo definition hablan siempre en terminos generales
of any party, then the functionality $\mathcal{F}_{auth}$ is deniable. Because a judge can distinguish
from the real protocol, the ideal functionality and the simulated ideal functionality.\\
Then we can define deniable protocols in a very similar way that a deniable authentication protocol is defined,
in terms of a judge, an informant and a misinformant.
Using the same arguments of \cite{conf/tcc/DodisKSW09} we can show  that a protocol is deniable if it
GUC-realize a fully simulatable functionality.

