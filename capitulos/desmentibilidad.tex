\chapter{Desmentibilidad}
La Desmentibilidad puede tener distintas acepciones, por ejemplo Encriptación
desmentible \cite{CanettiEtAl97}. Aquí nos referimos a la noción de Desmentibilidad asociada
a la Autentificación desmentible.\\
La Autentificación desmentible fue introducida por Dwork, Nahor y Sahai en \cite{DwoNaoSah04}.
Variadas modificaciones, generalizaciones e implementaciones han sido propuestas posteriormente,
como por ejemplo en \cite{journals/joc/RaimondoG09}. Aquí consideraremos y adaptaremos la definición
hecha por Dodis, Katz, Smith y Walfish en \cite{conf/tcc/DodisKSW09}, dado que esa definición
es la que aplica a una configuración concurrente y distribuida como la necesaria en este trabajo.\\
Intuitivamente decimos que un protocolo es desmentible si nadie puede probar a otro que una sesión
del protocolo, es decir un grupo específico de participantes con identidades públicas definidas,
se esta llevando a cabo o alguna vez se llevó a cabo. En \cite{conf/tcc/DodisKSW09} se muestra que
para el caso de la autentificación la desmentibilidad se puede obtener considerando un juez que
en forma \textit{online} debe decidir con quién esta hablando: un informante que esta observando
una sesión real del protocolo de autentificación, o un desinformante que no tiene acceso a la sesión
real del protocolo pero aún así quiere convencer al juez que la sesión se esta llevando a cabo.
El protocolo se dirá entonces que es un protocolo de autentificación desmentible online si
para todo juez y para todo informante existe un desinformante tal
que el juez no puede distinguir
si habla con el informante o el desinformante. En la versión completa de \cite{conf/tcc/DodisKSW09}
se demuestra se demuestra que esta noción es equivalente a GUC-realizar la funcionalidad ideal
$\mathcal{F}_{AUTH}$. Dodis y compañía señalan que en GUC un protocolo que realiza una funcionalidad
ideal $\mathcal{F}$ es tan desmentible como $\mathcal{F}$. La funcionalidad ideal $\mathcal{F}_{AUTH}$
es ``completamente simulable'', lo que significa que el protocolo puede ser simulado completamente sin
la participación de ningún participante de la sesión, luego la funcionalidad $\mathcal{F}_{AUTH}$
es desmentible.\\
De forma similar que en \cite{conf/tcc/DodisKSW09}, pero sin restringirnos a protocolos de autentificación,
definiremos la noción de desmentibilidad en línea o \textit{online}.

\section{Desmentibilidad online}

Consideraremos dos \textit{mundos}: el mundo real, donde el informante tiene acceso directo a una sesión
del protocolo analizado; y el mundo simulado, donde el desinformante no tiene acceso al protocolo.
Nos restringiremos a funcionalidades ideales, pues la desmentibilidad es una propiedad a ser chequeada en la
funcionalidad ideal.

\begin{definicion}[Mundo real]
Sea $\mathcal{F}$ una funcionalidad ideal que se ejecuta en el modo $\bar{\mathcal{G}}$-híbrido con participantes
$\tilde{P}_1, \ldots, \tilde{P}_n$, sea $\mathcal{D}$ el adversario \textit{dummy}, sea $\mathfrak{I}$ el
informante, $\mathcal{J}$ el juez y sea
$(V_\mathcal{F}, E_\mathcal{F}) =
\mathfrak{I}(
    \mathcal{H}(
        \mathcal{F},
        \mathcal{D},
        \tilde{P}_1,
        \ldots,
        \tilde{P}_n,
        \{\tilde{P_i}\}^{\bar{\mathcal{G}}}))$.
Definimos el mundo real como el grafo de ITMs $\mathfrak{R} = (V, E)$ donde:
$$V = \{\mathcal{J}, \bar{\mathcal{G}}\} \cup V_\mathcal{F}$$
$$E = \{(\mathcal{J}, \mathfrak{I})\} \cup E_\mathcal{F}$$
Denotamos por $\mathrm{Real}_{\mathcal{F}, \mathcal{J}, \mathfrak{I}}^{Den}$ a la variable aleatoria
que describe la salida de $\mathcal{J}$ al ejecutarse el grafo de ITMs $\mathfrak{R}$
\end{definicion}

\begin{definicion}[Mundo simulado]
Sea $\mathcal{F}$ una funcionalidad ideal que se ejecuta en el modo $\bar{\mathcal{G}}$-híbrido con participantes
$\tilde{P}_1, \ldots, \tilde{P}_n$, sea $\mathcal{D}$ el adversario \textit{dummy}, sea $\mathfrak{D}$ el
desinformante, $\mathcal{J}$ el juez.
Definimos el mundo simulado como el grafo de ITMs $\mathfrak{S} = (V, E)$ donde:
$$V = \{\mathcal{J}, \bar{\mathcal{G}}, \mathfrak{D}\}$$
$$E = \{(\mathcal{J}, \mathfrak{D}),
        (\mathcal{J}, \bar{\mathcal{G}}),
        (\mathfrak{D}, \bar{\mathcal{G}})\}$$
Denotamos por $\mathrm{Sim}_{\mathcal{F}, \mathcal{J}, \mathfrak{D}}^{Den}$ a la variable aleatoria
que describe la salida de $\mathcal{J}$ al ejecutarse el grafo de ITMs $\mathfrak{S}$
\end{definicion}

Definimos la desmentibilidad como la incapacidad del juez para distinguir entre una ejecución
real informada por el informante de la funcionalidad de una ejecución simulada por el desinformante.

\begin{definicion}[Desmentibilidad]
Decimos que una funcionalidad $\mathcal{F}$ es
desmentible si para todo juez $\mathcal{J}$ y todo informante
$\mathfrak{I}$ existe un desinformante $\mathfrak{D}$ talque:
$$\mathrm{Real}_{\mathcal{F}, \mathcal{J}, \mathfrak{I}}^{Den} \approx
\mathrm{Sim}_{\mathcal{F}, \mathcal{J}, \mathfrak{D}}^{Den}$$
\end{definicion}

Notamos que el experimento $\mathrm{Real}_{\mathcal{F}, \mathcal{J}, \mathfrak{I}}^{Den}$ es una transformación
sintáctica de la ejecución en UC de la funcionalidad ideal $\mathcal{F}$, como lo demuestra el siguiente teorema.

\begin{teorema}
Para toda funcionalidad $\mathcal{F}$ ejecutada en el modelo $\bar{\mathcal{G}}$-híbrido se tiene que para
todo juez $\mathcal{J}$ y todo informante $\mathfrak{I}$ existe un ambiente $\mathcal{Z}$ talque
$$\mathrm{Real}_{\mathcal{F}, \mathcal{J}, \mathfrak{I}}^{Den} \approx
\mathcal{Z}(\mathcal{H}(
        \mathcal{F},
        \mathcal{D},
        \tilde{P}_1,
        \ldots,
        \tilde{P}_n,
        \{\tilde{P}_i\}^{\bar{\mathcal{G}}}))
$$
y también para todo ambiente $\mathcal{Z}$ existe un juez $\mathcal{J}$ y un informante $\mathfrak{I}$
tales que

$$\mathrm{Real}_{\mathcal{F}, \mathcal{J}, \mathfrak{I}}^{Den} \approx
\mathcal{Z}(\mathcal{H}(
        \mathcal{F},
        \mathcal{D},
        \tilde{P}_1,
        \ldots,
        \tilde{P}_n,
        \{\tilde{P}_i\}^{\bar{\mathcal{G}}}))
$$

\label{teo:den}
\end{teorema}

\begin{proof}
\textit{(Teorema \ref{teo:den})}\\
En efecto, cualquier combinación juez-informante puede ser simulada por un ambiente $\mathcal{Z}$,
lo que prueba la segunda afirmación de teorema.\\
Si consideramos un informante $\mathfrak{I}$ que solo comunica al juez $\mathcal{J}$ con los participantes
y el adversario dummy $\mathcal{D}$, el juez $\mathcal{J}$ es capaz de simular internamente a cualquier
ambiente $\mathcal{Z}$
pues $\mathfrak{I}$ le provee una interfaz con vista idéntica a la vista de $\mathcal{Z}$. Lo que nos
permite probar la primera afirmación del teorema.
\end{proof}

Como corolario del teorema \ref{teo:den} tenemos el siguiente resultado
\begin{corolario}
Sea $\pi$ protocolo que GUC-emula a una funcionalidad
ideal $\mathcal{F}$ desmentible. Entonces $\pi$ es desmentible.
\label{cor:den}
\end{corolario}

\begin{proof}
La demostración es directa dado que el experimento real para $\pi$ es identico a ejecutarlo en GUC.
Como $\pi$ GUC-emula a $\mathcal{F}$ existe un simulador que simula a $\pi$ con $\mathcal{F}$, el cual
puede ser usado para simular $\pi$ con el desinformate que existe para $\mathcal{F}$.
\end{proof}
