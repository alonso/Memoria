\section{Resumen Ejecutivo}

El problema de comunicación anónima autentificada consiste en diseñar un protocolo
que permita intercambiar mensajes entre un conjunto de participantes, de forma tal
que cada emisor de un mensaje determinia el destinatario de su
mensaje y, una vez que se envía el mensaje,  este es efectivamente recibido por el destinatario
determinado. La información que revela el protocolo en su ejecución debe mantener el
anonimato, es decir debe ser tal que no permite a ningún adversario determinar 
información relacionada a las identidades  de los participantes.
El protocolo debe permitir a cada destinatario determinar con exactitud quién es el autor
de cada mensaje que recibe, sin que esto contradiga el anonimato.
Adicionalmente el protocolo debe mantener las garantías anteriores inclusive si es ejecutado en
un ambiente concurrente, es decir es ejecutado con indeterminados otros protocolos.\\
Las aplicaciones de la comunicación anónima autentificada son variadas. Por ejemplo es útil
para diseñar sistemas de denuncia anónima de delitos donde adicionalmente se desea discriminar
la información recibida segun la identidad del enviador. Esto puede ser útil si algunos
informantes son más creibles que otros.\\
En este trabajo se plantea el problema de comunicación anónima autentificada y se
demuestra constructivamente la existencia de un protocolo que resuelve dicho problema.
Para ello se estudian tópicos avanzados de Criptografía como \textit{Universal
Compossability}, \textit{Generalized Universal Composability}, Anonimato,
Desmentibilidad y las distintas primitivas criptográficas asociadas a dichos tópicos.
Se definen rigurosamente las propiedades que debe tener un protocolo para
resolver el problema planteado.
Finalmente se diseña un protocolo eficiente para el cual se puede garantizar
matematicamente que satisface las propiedades necesarias para resolver el problema
de comunicación anónima autentificada.
