\chapter{Anonimato} \label{sect:AC}
% Para que quede mejor podria poner tu definicion de anonimato, pero consecuentemente deberia probar
% que la funcionalidad ideal F_aac cumple con una de las defs. Cosa que no he hecho
Anonymous channels allow users to exchange messages without revealing their identities. Several protocols
have been proposed in the literature for anonymous channels. The modern study of anonymous channels was
started in \cite{journals/cacm/Chaum81} with \textit{mix-nets}. In a mix-net protocol the vector of all
parties encrypted messages are sent trough a set of \textit{mixers}. Each mixer perform an operation on cyphertexts
(usually partial decryption or reencryption) and send a random permutation to the next mixer. Finally the
last mixer publish a permutation of the vector of parties messages. Several modifications have been proposed
to mix-nets since Chaum's seminal paper, increasing tolerance to dishonest parties, robustness and many other
desirable properties.\\
To realize our protocol we use the universal composable mix net proposed by Wikstr\"om in \cite{Wikstrom04a}.
Basically Wikstr\"om's mix net proceeds as follows:

\begin{enumerate}

\item Each sender $P_i$ waits for mixers public keys and computes the product public key.
      Then each encrypt his message under the product public key, publish the cyphertext
      to a bulletin board an prove in zero knowledge that it is a valid cyphertext.
\item Each mix net $M_j$ $j\in{1, \ldots, k}$ discards all the published cyphertexts that
      are not valid. Then, for $l = 1, \ldots, k$ if $l = j$ the mixer partially decrypt
      the list of cyphertexts obtained from the bulletin board, perform a randomly chosen
      permutation on the list of cyphertext, publish on the bulletin board and prove in
      zero knowledge that the published list is a random permutation of the previous list.
      If $l \neq j$ the mixer must check that the permutation published by the mixer $M_l$
      is a valid one. Finally lexicographically sort the final published list and output it.

\end{enumerate}

In \cite{Wikstrom04a} is shown that this protocol UC-realize the ideal functionality $\mathcal{F}_{MN}$,
defined in figure \ref{func:F_MN}, in the $\mathcal{F}_{KG}-hybrid$ model. 
% Podria ser bueno In the apendix we show that this protocol also GUC-realize con F_KG share


\begin{figure}
\begin{centering}
\framebox{\begin{minipage}[t]{1\columnwidth}
\center{The Ideal functionality $\mathcal{F}_{MN}$ running with mixers $M_{1}, \ldots, M_{k}$, senders
        $P_{1}, \ldots, P_{N}$, and ideal adversary $\mathcal{S}$}
\begin{enumerate}
    \item Initialize a list $L = \emptyset$, and sets $J_P = \emptyset$ and $J_M = \emptyset$.
    \item Suppose $(P_{i}, \mathtt{Send}, m_{i})$  $m_{i} \in G_q$ is received from $\mathcal{C_I}$.
          If $i\notin J_P$, set $J_P \leftarrow J_P \cup \{i\}$, and append $m_i$ to the list $L$. Then
          hand $(\mathcal{S}, P_i, \mathtt{Send})$ to $\mathcal{C_I}$.
    \item Suppose $(M_{j}, \mathtt{Run})$ is received from $\mathcal{C_I}$. Set
          $J_M \leftarrow J_M \cup \{j\}$. If $|J_M | \geq k/2$, then sort the list $L$ lexicographically
          to form a list $L'$, and hand
          $((\mathcal{S}, M_{j}, \mathtt{Output}, L'), \{M_l , \mathtt{Output}, L'\}_{l=1}^{k})$ to
          to $\mathcal{C_I}$. Otherwise, hand $\mathcal{C_I}$ the list $(\mathcal{S}, M_{j}, \mathtt{Run})$
\end{enumerate}
\end{minipage}}
\end{centering}
\caption{The functionality $\mathcal{F}_{MN}$}
\label{func:F_MN}
\end{figure}

