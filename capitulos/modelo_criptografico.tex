\chapter{Modelo Criptográfico}

Demostraremos las garantías de seguridad de nuestro protocolo en el
\textit{framework Generalized Universal Composability} (GUC). GUC \cite{conf/tcc/CanettiDPW07}
es una generalizacion del \textit{framework Universal Composability}
\cite{conf/focs/Canetti01}. Ambos sirven para modelar protocolos criptgráficos concurrentes,
pero GUC adicionalmente modela protocolos concurrentes que comparten estado entre si.
Primero revisaremos el framework UC pues GUC se construye a partir de UC con pequeñas,
pero muy significativas, modificaciones.\\

\section{The Universal Composability framework (UC)}
UC es una metodología para modelar modularizadamente protocolos criptograficos que son ejecutados en redes
del ``mundo real" (por ejemplo internet). El espiritu de UC es diseñar un protocolo y luego abstraer la
seguridad que uno espera de él en un protocolo ideal, ejecutado en condiciones especiales que garantizan
su seguridad. Luego se debe demostrar que ejecutar el protcolo real y ejecutar el protocolo ideal es
escencialmente lo mismo, por lo tanto el protocolo real es tan seguro como el protocolo ideal.\\
Los protocolos reales son pueden ser ejecutados concurrentemente con muchos otros protocolos,
y tambien pueden ser ejecutados distribuidos entre varios participantes.
Como en el mundo real el protocolo puede ser monitoreado por terceros y algunos participantes pueden salirse
arbitrareamente del protocolo atentando con la seguridad. Todo el posible mal comoportamiento es ejecutado por una sola
máquina, el Adversario (real) $\mathcal{A}$. En una ejecución del protocolo el adversario puede espíar y
manipular todos los mensajes intercambiados, también puede manejar la distribución de mensajes a su antojo,
y además puede \textit{corromper} participantes del protocolo y ejecutar codigo arbitrario en ellos.\\
Por otro lado estan los protocolos ideales, llamados \textit{funcionalidades ideales} denotadas por 
$\mathcal{F}$. Las funcionlidades ideales son ejecutadas en un mundo ideal, donde es una entidad confiable
la encargada de ejecutar su código. En el mundo ideal existe un adversario ideal o \textit{simulador}
$\mathcal{S}$, pero este no puede espíar ni controlar la comunicación más de lo que la funcionalidad ideal
permite.\\
La configuración en la que el protocolo real es ejecutado con el adversario real es conocida como \textit{mundo
real}, y la configuración en que la funcionalidad ideal es ejecutada con el adversario ideal es conocida como
\textit{mundo ideal}. En ambos mundos la ejecución del protocolo concurrentemente con otros protocolos esta a cargo
de una máquina especial conocida como el ambiente y denotado por $\mathcal{Z}$. De este modo en UC los protocolos
pueden ser analizados aislados de resto del mundo. Para asegurarse que el protcolo real alcanza la seguridad deseada
se debe tener que para cualquier ejecución del protocolo en el mundo real y para cualquier estrategia adversarial
(esto es para todo ambiente y para todo adversario) existe una estrategia adversarial con recursos limitados (un
adversario ideal) que tiene el mismo efecto que la estrategia adversarial real (el ambiento no es capaz de percatarse
de ninguna diferencia entre la ejecución del protocolo real y el protocolo ideal).\\
Para definir formalmente las nocienes intuitivas descritas anteriormente es necesario introducir un modelo de
cálculo conocido como Máquinda de Turing Interactiva , más precisamente sitemas de ITM.

\subsection{Sistemas de Máquinas de Turing Interactivas}

Las Máquinas de Turing Interactivas corresponden a Máquinas de Turing con cintas especiales que pueden ser
escritas externamente. A dichas cintas las llamamos escribibles externamente (EW), y son de escritura única,
es decir, el cabezal siempre se mueve a la derecha.

\newtheorem{definicion}{Definición}

\begin{definicion}
Una Máquina de Turing Interactiva $M$ es una Máquina de Turing con las siguientes cintas:
\begin{enumerate}
    \item Una cinta EW de identidad.
    \item Una cinta EW del parámetro de seguridad.
    \item Una cinta EW de entrada.
    \item Una cinta EW de comunicación entrante.
    \item Una cinta EW de salidas de subrutinas.
    \item Una cinta de salida.
    \item Una cinta de bits aleatorios.
    \item Una cinta de activación, de lectura y escritura y de 1 bit de tamaño.
    \item Una cinta de lectura y escritura para trabajo. 
\end{enumerate}
\end{definicion}

La cinta de identidad contiene un string que representa la identidad de $M$, que se interpreta
como si estuviera compuesto por dos substrings: el identificadod de sesión (SID) y el identificador
de participante (PID). Identificamos a cada instancia de una ITM (ITI) por el par
$\mu = (\langle M \rangle, id)$, con $\langle M \rangle$ el código de $M$ y $id$ el contenido de la cinta
de identidad. En general omitimos los $\langle \rangle$ y con $M$ nos referimos tanto a la máquina como
al código.\\
La cinta del parámetro de seguridad contiene un string de la forma $1^k$, con $k$ el parámetro de seguridad 
\footnote{El parámetro de seguridad indica el nivel de seguridad en el cual se esta ejecutando
la máquina, y en general mientras crece se debería tener que la seguridad del protocolo ejecutado
con la máquina también crece.}.\\
La cinta de salida contendrá la salida de $M$ una vez que haya terminado.\\
La cinta de bits aleatorios contiene suficientes bits aleatorios para que $M$ pueda realizar sus cálculos.\\
La cinta de trabajo es la usual cinta de trabajo de las Máquinas de Turing.\\
La cinta de activación tiene el valor 0 si la $M$ no esta activada y 1 si lo esta. La secuencia de
configuraciones de una ejecución de $M$
\footnote{Una configuración corresponde a un objeto que determina completamente un instante
en la computación de una MT. Podemos ver la ejecuciónde una MT como una secuencía de configuraciones, donde la
primera configuración corresponde a la MT en su estado inicial y la(s) cinta(s) con la(s) entrada(s), y la
configuración final corresponde a la MT en un estado final.}
esta compuesta por subsecuencias en las
que en cada configuración la $M$ esta activada. A dichas subsecuencias se les conoce como \textit{secuencias
de activación}.\\
Las otras cintas toman importancia cuando $M$ es ejecutada ``conectada" con otras máquina
en un Sistema de ITMs.\\

\begin{definicion}
Un Sistema de ITMs $S$ viene dado por $S = (I, C)$, donde $I$ es la ITM inicial y $C$ es la función
de control $C:\{0,1\}^* \to \{0,1\}$.
\end{definicion}
Inicialmente la ITI $\mu_0 = (I, 0)$ es activada, el sITM terminará cuando $I$ termine y su
salida sera la salida lo que $I$ deje en su cinta de salida.\\
Una ITI $\mu = (M, id)$ puede escribir en una de las cintas de otra ITI $\mu' = (M', id')$, 
para ello es necesario que $\mu$ ejecute una instrucción
especial llamada \texttt{escritura-externa} y debe especificar la cinta en de $\mu'$ en la que quiere
escribir y los datos a escribir en esa cinta
\footnote{Formalmente podríamos decir que $M$ entra en un estado especial y en su cinta de
trabajo se encuentra un string $x$ que determina $\mu'$, la cinta objetivo y los datos a escribir
en la cinta objetivo}.
La semántica de la instrucción de \texttt{escritura-externa} es como sigue:\\

\begin{enumerate}
    \item Si la funciónde control $C$ aplicada a toda la secuencia de intrucciones \texttt{escritura-externa}
          que se han realizado hasta ahora retorna 0, entonces la instrucción es ignorada.
    \item Si $C$ retorna 1 pero no existe una ITI en el sITM $\mu'' = (M'', id'')$ talque $id''=id'$, se
          crea una nueva ITI con código $M'$ y con identidad $id'$. Para ello se crea una nueva ITM
          que en la cinta de identidad contiene $id'$, el la cinta del parámetro de sugirdad contiene $1^k$
          y en la cinta de bits aleatorios contiene suficientes bits aleatorios. A continuación se evalúa
          el punto siguiente.
    \item Si $C$ retorna 1 y existe una ITI en el sITM $\mu'' = (M'', id'')$ talque $id''=id'$:
    \begin{enumerate}
        \item Si la cinta objetivo de la intrucción era la cinta de comunicación entrante de $\mu'$, entonces
              los datos especificados son escritos en la cinta de comunicación entrante de $\mu''$ y $\mu''$
              es activada. Notemos que esto es hecho independiente de si el código de $\mu''$ es el mismo
              código especificado por $\mu$, con el fin de rescatar que una ITI no conoce el código de la
              ITI con que se comunica a través de escrituras en la cinta de comunicación entrante.
        \item Si la cinta objetivo era la cinta de entrada de $\mu'$ y $M' = M''$, entonces los datos
              especificados son escritos en la cinta de entrada de $\mu''$ y $\mu''$ es activada. En este caso
              la instrucción modela llamados a otras ITIs como subrutina, dentro de un entorno seguro ($\mu$
              conoce el código que esta ejecutando $\mu''$).
        \item Si la cinta objetivo es la cinta de salida de subrutina de $\mu'$, entonces los datos
              especificados son escritos en la cinta de salida de subrutina de $\mu''$. En este caso la
              instrucción modela el retorno de una llamada a subrutina, en que la subrutina no conoce
              el código de la ITI que la llamó.
    \end{enumerate}
\end{enumerate}

Ahora estamos listos para definir la ejecución de un protocolo en UC

\subsection{Ejecución de un protocolo}

\section{Generalized Universal Composabillity}

Before the GUC framework were proposed an alternative way
to model protocols sharing state in the UC framework was the use of JUC theorem, but this is not as general as
it only allows protocols sharing state among themselves. Instead the GUC framework models protocols sharing state with
unpredictable other protocols.\\

