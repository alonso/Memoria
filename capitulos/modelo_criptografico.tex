\chapter{Modelo Criptográfico}

Demostraremos las garantías de seguridad de nuestro protocolo en el
\textit{framework Generalized Universal Composability} (GUC). GUC \cite{conf/tcc/CanettiDPW07}
es una generalizacion del \textit{framework Universal Composability}
\cite{conf/focs/Canetti01}. Ambos sirven para modelar protocolos criptgráficos concurrentes,
pero GUC adicionalmente modela protocolos concurrentes que comparten estado entre si.
Primero revisaremos el framework UC pues GUC se construye a partir de UC con pequeñas,
pero muy significativas, modificaciones.\\

\subsection{The Universal Composability framework (UC)}
UC es una metodología para modelar modularizadamente protocolos criptograficos que son ejecutados en redes
del ``mundo real" (por ejemplo internet). El espiritu de UC es diseñar un protocolo y luego abstraer la
seguridad que uno espera de él en un protocolo ideal, ejecutado en condiciones especiales que garantizan
su seguridad. Luego se debe demostrar que ejecutar el protcolo real y ejecutar el protocolo ideal es
escencialmente lo mismo, por lo tanto el protocolo real es tan seguro como el protocolo ideal.\\
Los protocolos reales son pueden ser ejecutados concurrentemente con muchos otros protocolos,
y tambien pueden ser ejecutados distribuidos entre varios participantes.
Como en el mundo real el protocolo puede ser monitoreado por terceros y algunos participantes pueden salirse
arbitrareamente del protocolo atentando con la seguridad. Todo el posible mal comoportamiento es ejecutado por una sola
máquina, el Adversario (real) $\mathcal{A}$. En una ejecución del protocolo el adversario puede espíar y
manipular todos los mensajes intercambiados, también puede manejar la distribución de mensajes a su antojo,
y además puede \textit{corromper} participantes del protocolo y ejecutar codigo arbitrario en ellos.\\
Por otro lado estan los protocolos ideales, llamados \textit{funcionalidades ideales} denotadas por 
$\mathcal{F}$. Las funcionlidades ideales son ejecutadas en un mundo ideal, donde es una entidad confiable
la encargada de ejecutar su código. En el mundo ideal existe un adversario ideal o \textit{simulador}
$\mathcal{S}$, pero este no puede espíar ni controlar la comunicación más de lo que la funcionalidad ideal
permite.\\
La configuración en la que el protocolo real es ejecutado con el adversario real es conocida como \textit{mundo
real}, y la configuración en que la funcionalidad ideal es ejecutada con el adversario ideal es conocida como
\textit{mundo ideal}. En ambos mundos la ejecución del protocolo concurrentemente con otros protocolos esta a cargo
de una máquina especial conocida como el ambiente y denotado por $\mathcal{Z}$. De este modo en UC los protocolos
pueden ser analizados aislados de resto del mundo. Para asegurarse que el protcolo real alcanza la seguridad deseada
se debe tener que para cualquier ejecución del protocolo en el mundo real y para cualquier estrategia adversarial
(esto es para todo ambiente y para todo adversario) existe una estrategia adversarial con recursos limitados (un
adversario ideal) que tiene el mismo efecto que la estrategia adversarial real (el ambiento no es capaz de percatarse
de ninguna diferencia entre la ejecución del protocolo real y el protocolo ideal).\\
Para definir formalmente las nocienes intuitivas descritas anteriormente es necesario introducir un modelo de
cálculo conocido como Máquinda de Turing Interactiva , más precisamente sitemas de ITM.

\subsection{Sistemas de Máquinas de Turing Interactivas}
Las Máquinas de Turing Interactivas corresponden a Máquinas de Turing multicinta que poseen cintas que pueden ser
escritas externamente. D

\section{Generalized Universal Composabillity}

Before the GUC framework were proposed an alternative way
to model protocols sharing state in the UC framework was the use of JUC theorem, but this is not as general as
it only allows protocols sharing state among themselves. Instead the GUC framework models protocols sharing state with
unpredictable other protocols.\\

